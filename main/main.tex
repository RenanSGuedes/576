\documentclass[a4paper, 12pt]{article}
\usepackage{config}

\usepackage{import}

\begin{document}
	\section{Introdução}
	

	
	\section{Objetivos}
	\section{Materiais e Métodos}
	\section{Resultados e discussão}
	
	A partir dos gráficos obtidos, nota-se que ao elevar a velocidade de deformação dos corpos de prova cilíndricos ocorre elevação nos valores de módulo de elasticidade. Na primeira parte do experimento, sob velocidade de \SI{.5}{\milli\meter/\second}, foi obtido um valor de $\textrm{E}_{1}=\SI{3.22}{\mega\pascal}$ com base no ajuste realizado. Na segunda etapa, a \SI{.8}{\milli\meter/\second} o comportamento mecânico da batata inglesa ocasionou aumento de \SI{.32}{\mega\pascal} ($\textrm{E}_{2}=\SI{3.54}{\mega\pascal}$) e, por fim, na bateria final, o valor de $\textrm{E}$ foi novamente incrementado levando a um $\textrm{E}_{3}=\SI{3.77}{\mega\pascal}$. 
	
	Com base
	
	\section{Conclusão}

\end{document}