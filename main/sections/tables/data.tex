\begin{table}[h!]
	\centering
	\begin{tabular}{|r|r|r|r|}
		\hline
		\multicolumn{1}{|c|}{\textbf{\begin{tabular}[c]{@{}c@{}}Corpo \\ de Prova\end{tabular}}} & \multicolumn{1}{c|}{\textbf{Diâmetro (\SI{}{\milli\meter})}} & \multicolumn{1}{c|}{\textbf{Altura (\SI{}{\milli\meter})}} & \multicolumn{1}{c|}{\textbf{Velocidade (\SI{}{\milli\meter/\second})}} \\ \hline
		1 & 13.16 & 30.83 & 0.5 \\ \hline
		2 & 12.89 & 31.34 & 0.5 \\ \hline
		3 & 13.20 & 30.85 & 0.5 \\ \hline
		4 & 13.25 & 31.80 & 0.5 \\ \hline
		5 & 13.15 & 32.25 & 0.5 \\ \hline
		6 & 12.81 & 30.75 & 0.8 \\ \hline
		7 & 12.89 & 31.78 & 0.8 \\ \hline
		8 & 13.44 & 31.14 & 0.8 \\ \hline
		9 & 13.60 & 30.91 & 0.8 \\ \hline
		10 & 13.52 & 31.03 & 0.8 \\ \hline
		11 & 13.15 & 31.50 & 1.2 \\ \hline
		12 & 13.35 & 30.70 & 1.2 \\ \hline
		13 & 13.53 & 30.71 & 1.2 \\ \hline
		14 & 13.32 & 31.40 & 1.2 \\ \hline
		15 & 13.28 & 32.51 & 1.2 \\ \hline
	\end{tabular}
	\caption{Dimensão dos corpos de prova submetidos aos testes de compressão para três valores de velocidade}
	\label{tab:my-table}
\end{table}